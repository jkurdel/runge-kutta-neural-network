% Great thanks for Cezary Sałbut (this latex file is mainly based on his work)

\documentclass[a4paper,12pt,fleqn]{article}


\usepackage[utf8]{inputenc} 
\usepackage{polski}
\usepackage[polish]{babel}
\usepackage[pdftex]{color,graphicx}
\usepackage{subfig}
\usepackage{indentfirst}
\usepackage{booktabs}
\usepackage{tabularx}
\usepackage{multirow}
\usepackage{pdflscape}
\usepackage{array}
\usepackage{amsmath}
\usepackage{appendix}
\usepackage{hyperref}
\hypersetup{
    colorlinks,%
    citecolor=black,%
    filecolor=black,%
    linkcolor=black,%
    urlcolor=blue
}

\usepackage{listings}
\lstset{
	numbers=left, 
	stepnumber=1, 
	basicstyle={\footnotesize \ttfamily},
	language=C, 
	captionpos=b,
	xleftmargin=6mm,
	aboveskip=\medskipamount,
	belowskip=\bigskipamount,
	tabsize=4
}

\usepackage{title-page}

\usepackage[top=28mm, bottom=30mm, left=21mm, right=21mm]{geometry}

\linespread{1.1}		%interlinia

\usepackage{boxedminipage} 
%% change the below lengths to suit your needs: 
\setlength{\fboxrule}{1pt} 
\setlength{\fboxsep}{5pt}



\renewcommand{\appendixtocname}{Dodatki}
\renewcommand{\appendixpagename}{Dodatki}


\widowpenalty=10000
\clubpenalty=10000


\newcommand{\autor}{Jan Kurdel}
\newcommand{\tytulpl}{Identyfikacja dynamiki obiektu latającego metodą najmniejszych kwadratów z ortogonalizacją}
%\newcommand{\tytulen}{Switched--mode power supply with programmable output voltage}
\newcommand{\uczelnia}{POLITECHNIKA WARSZAWSKA}
\newcommand{\wydzial}{Wydział Elektroniki i Technik Informacyjnych}
\newcommand{\instytut}{Instytut Systemów Elektronicznych}
\newcommand{\promotor}{dr inż. Stanisława Jankowskiego}
\newcommand{\praca}{Praca inżynierska}
\newcommand{\miejscerok}{Warszawa, 2012}
\newcommand{\indeks}{214460}

\title{\tytulpl}
\author{\autor}